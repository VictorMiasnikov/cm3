%%
%% $Id$
%%

\chapter{MScheme Reference Manual}

The MScheme system follows the {\it Revised$^4$ Report on the
  Algorithmic Language Scheme (R4RS)\/}, and therefore a detailed description
of the Scheme language will be omitted from the present document.  We will
present only the primitives provided by MScheme and also the differences
with respect to {\it R4RS\/}.

\section{Scheme Primitives}

\subsection{Basic set}

\subsection{Modula-Scheme extended set}

\section{MScheme Module System}\label{sec:modules}

MScheme provides, by default, a simple module system that exports a single
Scheme procedure.
\begin{verbatim}
(require-modules mod1 mod2 ... modN)
\end{verbatim}
performs the following actions: for each of {\tt mod1} to {\tt modN},
if that pathname or module name (see section~\ref{sec:provided}) has not
previously been loaded, it is loaded (in order).  

The system guarantees that ``loading loops'' cannot happen, since each
file will only be loaded once.

Modula-3 programmers using MScheme-provided interpreters directly
should be aware that in order for the module system to be available,
the string {\tt require} should be passed as the first filename for
the created interpreter.  (A {\tt Scheme.T} does not, for example,
provide {\tt require-modules} by default.)

\noindent{\em Example.}
\begin{verbatim}
(require-modules "mergesort" "../myfile.scm")
\end{verbatim}
Load in the {\tt mergesort} provided library and the Scheme source file
{\tt ../myfile.scm}\ .  Actual file paths are relative to the current working directory of the Scheme interpreter process.




\section{R$^5$RS Hygienic Macros}

The {\tt mbe} module provides the ``hygienic'' macros from the {\it
  Revised$^5$ Report on the Algorithmic Language Scheme (R5RS)\/}
(this module also works in JScheme, and uses JScheme/MScheme primitive
macros to implement the {\it R5RS} macros).

\section{Modula-3 Extension Library}

\section{Modula-3 Stub Generator and Scheme Bindings}

\section{Navigable Scheme Environments}

\section{Synchronized Scheme Environments}

\section{Implementation Restrictions}

The implementation restrictions are currently the same as for JScheme (or at least very similar).

\begin{itemize}

\item Only a single numerical data type is provided, double precision
  as provided by Modula-3.  This means IEEE~754 double precision on
  most architectures.  MScheme does not properly distinguish between
  {\em exact\/} and {\em inexact\/} arithmetic.  There are probably some
  bugs where primitives are implemented with Modula-3's {\tt TRUNC} function,
  which might lead to an incorrect array index being calculated owing to
  rounding errors 
  (JScheme would have the same bugs).

\item {\tt call-with-current-continuation} only unwinds the stack.

\section{Source Description}

\section{Provided Scheme Libraries (Subject to Change)}\label{sec:provided}

General-purpose Scheme libraries are provided in the source directory
{\tt t/mscheme/scheme-lib/src}.  Libraries in this directory are {\em
  bundled\/} into the Scheme executable using the {\tt m3bundle}
program.  Hence they are available even without access to the
filesystem and are loaded by providing the name of the library without
the extension, e.g., the library provided by the file {\tt
  t/mscheme/scheme-lib/src/exit.scm} can be loaded simply by running
the following Scheme command:
\begin{verbatim}
(load "exit")
\end{verbatim}
Similarly, these libraries can be referred to without their extensions
when using the {\tt require} module system (section~\ref{sec:modules}).

\end{itemize}

\section{Primitives}

\input src/prims

